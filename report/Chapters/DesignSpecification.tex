\chapter{Design \& Specification}

\section{Design}

As referenced in figure \ref{fig:model}, the ABM will consist of bank agents representing the market-makers and some different types of trading agents that are connected to it as well as some retail traders. Data is fed to the bank agents to determine an exchange rate and and this is observable to the traders who can then act on the rate and trade accordingly.

The model should hence produce a reasonable simulation of the forex market and with more trading activity a trend for the exchange rate for EUR/USD will be produced by the market-makers in the model. From this graph of exchange rate over time, I can compare it to historic data of exchange rates over time. I can then look at the two graphs and use statistical analysis such as the product moment correlation coefficient (PMCC) to determine the correlation between the historic data and the model's output.

\subsection{Model Design}
The model will most likely consist of a very basic environment for agents to exist in. It will be an array of agents in the model's scheduler. Agents should be activated in a random order for each model step to be fair in an ideal sense, however any trades that a bank makes or a bank's traders make will be executed first. This is to reflect the fact that the inter-bank market is a faster market primarily governed by algorithmic trading with around 92\% of trading in the FX market performed by trading algorithms. \cite{kissell2020algorithmic}

\subsection{Agent Design}
Agents will be given different trading functions dependent on what type of agent they are. Bank agents will prioritise minimising risk and will trade higher volumes at lower yields to service its clients. Trading agents such as the buy and sell agents belonging to the banks may want to see higher risk trading opportunities. Retail agents may take on even greater risk to match their higher profit incentive, and may also act in a random manner at times to attempt to mirror the human aspect of retail traders. All agents do not have to trade at every model time step, they can hold long positions if they choose.

\section{Specification}

\subsection{Development Environment}
The Mesa library for Python 3 was chosen as the development package of choice as it comes with ABM data collection built into the library as well as agent scheduling. This means that the basic structure of an agent based model can be built very quickly with some minimal boilerplate code. Furthermore, a front end web server is available for mode visualisation in the form of graphs and histograms etc. It also allows for custom data visualisation components to be built for the web server through JavaScript and allows for fully custom agent behaviour through each step of the model.

\subsection{Front End Specifications}
The Mesa library already provides a front end application for users to view the output of the model and step through each iteration. 
\begin{itemize}
\item The project should aim to provide a clear visual output from the model that shows the exchange rate against time on a line graph for each individual market maker in the model.
\item Options should be available for users to average the market-maker exchange rates and compare the model output to historic data available.
\item The front end should be able to display metrics about differences in average yield between different types of trading agents as well as profit distribution.
\end{itemize} 


\subsection{Agent Specifications}
\begin{itemize}
\item The Agent should provide intelligent mechanisms to use when trading with a market-maker.
\item The market-makers should have statistically sound and consistent internal functions for calculating the exchange rate at each time step from the given data.
\end{itemize}

\subsection{Model Specifications}
\begin{itemize}
\item The model should provide a suitable environment for agents to interact with one another, whether it be a grid or an array for the agents to reside in. The likely choice will be an array for agents to be placed into as movement does not need to be modelled.
\item The model should provide an appropriately fair activation schedule for the agents to accurately simulate the FX market. Random activation on the schedule for each model step seems appropriately fair for all agents in the model. A distribution for traders trading with the bank to be activated first may be a more realistic approach and will be considered.
\item The model should provide data collection tools to produce a model output to be visualised.
\end{itemize}
